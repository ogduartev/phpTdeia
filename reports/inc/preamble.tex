\usepackage[spanish,es-nodecimaldot]{babel}
\usepackage[utf8x]{inputenc}
\usepackage[T1]{fontenc} 
\usepackage{lmodern}
\usepackage{amsmath}
\usepackage{amsfonts}
\usepackage{amssymb}
\usepackage{graphicx}
\usepackage{wrapfig}
% comentar para edición tipo libro
\usepackage[letterpaper,top=3.0cm,left=3cm,right=3cm,bottom=2.5cm]{geometry}
\usepackage{typearea}
\usepackage{pstricks,pst-all,pst-circ,pst-3dplot,pst-grad}
\usepackage{listings}
\usepackage[colorlinks=true,linkcolor=blue,citecolor=blue,urlcolor=blue,filecolor=blue]{hyperref}
\usepackage{longtable}
\usepackage{multirow}
\usepackage{multicol}
\usepackage{framed}
\usepackage{mdframed}
\usepackage[hyperref,framed]{ntheorem}
\usepackage{array}
\usepackage{verbatim}
\usepackage{fancyvrb}
\usepackage{chapterbib}
\usepackage{ifthen}
\usepackage{calc}
\usepackage{lscape}
\usepackage{rotating}
\usepackage{url}
\usepackage{enumerate}
\usepackage{subfigure}
\usepackage{rotating}
\usepackage{colortbl}
\usepackage{draftwatermark}
\usepackage{paralist}
\usepackage{thm-restate}
\usepackage[maxfloats=25]{morefloats}


\usepackage{hyperref}
\makeatletter
\newcommand*{\MyUrl}{\begingroup\@makeother\#\@MyUrl}
\newcommand*{\@MyUrl}[1]{%
  \href{#1}{$\circlearrowright$}%
  \endgroup}
\makeatother


%\SetWatermarkText{\includegraphics{figs/escudo}}
%\SetWatermarkScale{1.0}
%\SetWatermarkAngle{0}
\SetWatermarkText{TDEIA}
\SetWatermarkScale{0.75}
\SetWatermarkAngle{45}

%%%%%%%%%%%%%%%  pandoc  %%%%%%%%%%%%%%%%%%%%%%%%%%%
\providecommand{\tightlist}{%
  \setlength{\itemsep}{0pt}\setlength{\parskip}{0pt}}
%%%%%%%%%%%%%%%%%%%%%%%%%%%%%%%%%%%%%%%%%%%%%%%

\newlength{\tempX}
\newlength{\tempY}
\newlength{\tempXX}
\newlength{\tempYY}
\newcommand{\tdeiaAdjustpagesize}[2]
{
  \setlength{\tempX}{#1}
  \setlength{\tempY}{#2}
  \setlength{\tempXX}{730mm}
  \setlength{\tempYY}{520mm}
  \ifthenelse{\tempX>\tempXX \OR \tempY>\tempYY}{\tdeiaNewpagesize{A0}}
  {
    \setlength{\tempXX}{520mm}
    \setlength{\tempYY}{340mm}
    \ifthenelse{\tempX>\tempXX \OR \tempY>\tempYY}{\tdeiaNewpagesize{A1}}
    {
      \setlength{\tempXX}{360mm}
      \setlength{\tempYY}{260mm}
      \ifthenelse{\tempX>\tempXX \OR \tempY>\tempYY}{\tdeiaNewpagesize{A2}}
      {
        \setlength{\tempXX}{176mm}
        \setlength{\tempYY}{135mm}
        \ifthenelse{\tempX>\tempXX \OR \tempY>\tempYY}{\tdeiaNewpagesize{A3}}
        {
          \tdeiaNewpagesize{letter}
        }
      }
    }
  }
}

\newcommand{\tdeiaNewpagesize}[1]
{
  \clearpage
  \KOMAoptions{paper=#1,pagesize}
  \recalctypearea
  \ifthenelse{\equal{#1}{A0}}
  {
    \newgeometry{height=1100mm,width=780mm,layoutvoffset=-550mm,layouthoffset=320mm,hcentering=true}
  }{}
  \ifthenelse{\equal{#1}{A1}}
  {
    \newgeometry{textheight=730mm,textwidth=520mm,layoutvoffset=-300mm,layouthoffset=200mm,hcentering=true}
  }{}
  \ifthenelse{\equal{#1}{A2}}
  {
    \newgeometry{textheight=520mm,textwidth=340mm,layoutvoffset=-180mm,layouthoffset=100mm,hcentering=true}
  }{}
  \ifthenelse{\equal{#1}{A3}}
  {
    \newgeometry{textheight=360mm,textwidth=260mm,layoutvoffset=-80mm,layouthoffset=40mm,hcentering=true}
  }{}
  \ifthenelse{\equal{#1}{letter}}
  {
    \restoregeometry
  }{}
}

%%%%%%%%%%%%%%%%%%%%%%%%%%%%%%%


\newlength{\tam}
\newlength{\tamA}
\newlength{\tamB}
\newlength{\tamC}
\newcommand{\tdeiaPaintSet}[1]
{
  \psline[linecolor=blue]#1
}
\newcommand{\tdeiaPaintVariable}[7]
{
  \scriptsize
  \sf
  \rput[bl](0.1,2.3){#1}
  \psline{->}(0,0)(5.2,0)
  \psline{->}(0,0)(0,2.2)
  \psline{-}(-.1,0.5)(.1,0.5)
  \psline{-}(-.1,1.0)(.1,1.0)
  \psline{-}(-.1,1.5)(.1,1.5)
  \psline{-}(-.1,2.0)(.1,2.0)
  \rput[r](-.2,0.0){$0.00$}
  \rput[r](-.2,0.5){$0.25$}
  \rput[r](-.2,1.0){$0.50$}
  \rput[r](-.2,1.5){$0.75$}
  \rput[r](-.2,2.0){$1.00$}
  \rput[t](0,-0.2){#2}
  \rput[t](1,-0.2){#3}
  \rput[t](2,-0.2){#4}
  \rput[t](3,-0.2){#5}
  \rput[t](4,-0.2){#6}
  \rput[t](5,-0.2){#7}
  \psset{yunit=2cm}
  \psset{xunit=5cm}
  \psline{-}(0.2,.05)(0.2,-.05)
  \psline{-}(0.4,.05)(0.4,-.05)
  \psline{-}(0.6,.05)(0.6,-.05)
  \psline{-}(0.8,.05)(0.8,-.05)
  \psline{-}(1.0,.05)(1.0,-.05)
}
\newcommand{\tdeiaPaintVariableTitle}[1]
{
  \scriptsize
  \sf
  \rput[tl](0.1,2.9){#1}
}
\newcommand{\tdeiaPaintLabel}[2]
{
  \psset{unit=1cm}
  \rput(6,2)
  {
    \psset{yunit=0.3cm}
    \rput[tl](0,-#1){\scriptsize{\sf #2}}
  }
}
\newcommand{\tdeiaPaintInputSet}[1]
{
  \psset{yunit=2cm}
  \psset{xunit=5cm}
  \psline[linecolor=red,fillstyle=hlines,hatchcolor=red]#1
}
\newcommand{\tdeiaPaintImportanceSet}[1]
{
  \psset{yunit=2cm}
  \psset{xunit=5cm}
  \psline[linecolor=brown,fillstyle=hlines,hatchcolor=brown]#1
}
\newcommand{\tdeiaPaintAggregationSet}[1]
{
  \psset{yunit=2cm}
  \psset{xunit=5cm}
  \psline[linecolor=black,fillstyle=hlines,hatchcolor=black]#1
}

\newenvironment{tdeiaDescriptionContent}{}{}

\newcommand{\tdeiaWriteEffectPropierty}[6]
{
 {
  \scriptsize
  \sf
  \begin{table}
   \centering
  \caption[Explicación de la propiedad `#2']{Explicación de la propiedad `#2' usada para evaluar cada efecto}
  \label{tab:effect_propierty_#1} 
  \begin{tabular}{|*{3}{p{4cm}|}} \hline
   \multicolumn{3}{|c|}{#2} \\ \hline
   {\em Sentido:} \ifthenelse{\equal{#3}{1}}{Creciente}{Decreciente} 
   & {\em Peso:} $#4$ & $\theta=#5$ \\ \hline 
   \multicolumn{3}{|p{12cm}|}{\em{Descripción:}} \\
   \multicolumn{3}{|p{12cm}|}{\tdeiaDescription{effect_propierties}{#1}} \\ \hline
   \multicolumn{3}{|p{12cm}|}{\rule{0cm}{0.25cm}} \\ %\hline
   \multicolumn{3}{|p{12cm}|}{\em{Variable lingüística}} \\ \hline %
   \tdeiaWriteMyVariable{#6}
  }
}

\newcommand{\tdeiaWriteVariable}[4]
{
   {\em Nombre:} #2 & {\em Mínimo:} $#3$ & {\em Máximo:} $#4$  \\ \hline 
   \multicolumn{3}{|p{12cm}|}{\em{Significado de las etiquetas:}} \\
   \multicolumn{3}{|p{12cm}|}{\tdeiaDescription{variables}{#1}} \\ \hline
   \multicolumn{3}{|c|}{\begin{pspicture}(0,0)(10,4)\rput[bl](1,1){\tdeiaPaintMyVariable{#1}}\end{pspicture}} \\ \hline
  \end{tabular}
  \end{table}
}

\newcommand{\tdeiaWriteMyEffectPropiertyInSummary}[5]
{
 #2 & 
 \ifthenelse{\equal{#3}{1}}{Creciente}{Decreciente}  & 
 #4 & \ref{tab:effect_propierty_#1} & \pageref{tab:effect_propierty_#1}  \\ \hline 
}

\newenvironment{tdeiaEffectPropiertiesSummary}
{
  \begin{table}
   \scriptsize
   \sf
   \centering
   \caption[Resumen de las propiedades usadas]{Resumen de las propiedades utilizadas para valorar la importancia de cada efecto}
   \label{tab:effect_propierties_summary}
   \begin{tabular}{|p{3cm}|p{2cm}|p{1.5cm}|p{1.5cm}|p{1.5cm}|}\hline
   Nombre & Sentido & Peso & Cuadro & Página \\ \hline 
}
{
   \end{tabular}
  \end{table}
}

\newcommand{\tdeiaWriteMyAggregatorInSummary}[3]
{
  #1 (Ver cuadro \ref{tab:aggregatorVar#3}) &
  \ifthenelse{\equal{#2}{simple_average}}{ Promedio simple & \tdeiaMethodologyAggregatorSimpleAverage{#3}}{}
  \ifthenelse{\equal{#2}{weighted_average}}{ Promedio ponderado & \tdeiaMethodologyAggregatorWeightedAverage{#3}}{}
  \ifthenelse{\equal{#2}{minimum}}{ Mínimo  & \tdeiaMethodologyAggregatorMinimum{#3}}{}
  \ifthenelse{\equal{#2}{maximum}}{ Máximo  & \tdeiaMethodologyAggregatorMaximum{#3}}{}
  \\ \hline
}

\newcommand{\tdeiaWriteAggregator}[2]
{
  \ifthenelse{\equal{#1}{linear_combination}}{\tdeiaMethodologyImportanceLinearCombination{#1}}{}
  \ifthenelse{\equal{#1}{simple_average}}{\tdeiaMethodologyAggregatorSimpleAverage{#1}}{}
  \ifthenelse{\equal{#1}{weighted_average}}{\tdeiaMethodologyAggregatorWeightedAverage{#1}}{}
  \ifthenelse{\equal{#1}{minimum}}{\tdeiaMethodologyAggregatorMinimum{#1}}{}
  \ifthenelse{\equal{#1}{maximum}}{\tdeiaMethodologyAggregatorMaximum{#1}}{}
}

\newenvironment{tdeiaAggregatorsSummary}
{
  \begin{table}
   \scriptsize
   \sf
   \centering
   \caption[Operadores de agregación]{Operadores de agregación (agregadores) empleados para estimar la importancia combinada $IM_c$ de $M$ efectos individuales, cada uno de importancia $IM_i$}
   \label{tab:aggregators_summary}
   \begin{tabular}{|p{3cm}|p{2cm}|p{8cm}|}\hline
   Nombre & Ecuación & Explicación \\ \hline 
}
{
   \end{tabular}
  \end{table}
}

\newcommand{\tdeiaWriteMyAggregatorInTableImp}[9]
{
  \begin{table}
    \centering
    \sf
    \scriptsize
    \caption{Variable Lingüística asociada a la '#2'}
    \label{tab:importanceVar} 
   \begin{tabular}{|p{10cm}|} \hline
   \begin{center}\textbf{#2}\end{center} \\ \hline
    \textit{Descripción:} \\ #4 \\ \hline 
    \textit{Significado de las etiquetas:} \\ #7 \\ \hline 
    \begin{pspicture}(0,0)(10,4)\rput[bl](1,1){\tdeiaPaintMyVariable{#5}}\end{pspicture} \\ \hline
   \end{tabular}
  \end{table}
}

\newcommand{\tdeiaWriteMyAggregatorInTableAgg}[9]
{
  \begin{table}
    \centering
    \sf
    \scriptsize
    \caption{Variable Lingüística asociada al agregador '#2'}
    \label{tab:aggregatorVar#1} 
   \begin{tabular}{|p{10cm}|} \hline
   \begin{center}\textbf{#2}\end{center} \\ \hline
    \textit{Descripción:} \\ #4 \\ \hline 
    \textit{Significado de las etiquetas:} \\ #7 \\ \hline 
    \begin{pspicture}(0,0)(10,4)\rput[bl](1,1){\tdeiaPaintMyVariable{#5}}\end{pspicture} \\ \hline
   \end{tabular}
  \end{table}
}

\newlength{\tdeiaFactorPositionX}
\newlength{\tdeiaFactorPositionDx}
\newlength{\tdeiaFactorPositionY}
\newlength{\tdeiaFactorPositionDy}
\newlength{\tdeiaTmp}

\newcommand{\tdeiaPaintFactorNode}[5]
{
  \setlength{\tdeiaFactorPositionDx}{1cm}
  \setlength{\tdeiaFactorPositionDy}{-0.6cm}
  
  \setlength{\tdeiaFactorPositionX}{#3\tdeiaFactorPositionDx}
  \rput[bl](\tdeiaFactorPositionX,\tdeiaFactorPositionY)
  {
    \begin{tabular}{|p{4cm}|p{0.75cm}|p{0.75cm}|}\hline
     \rule{0cm}{0.3cm}\cellcolor{gray!25} \hyperref[sec:factor#1]{#2} & \cellcolor{gray!50}#4 &\cellcolor{gray!50} #5 \\ \hline
    \end{tabular}
  }
 
  \setlength{\tdeiaTmp}{\tdeiaFactorPositionX}
  \addtolength{\tdeiaTmp}{0.3cm}
  \pnode(\tdeiaTmp,\tdeiaFactorPositionY){Fo#1}

  \setlength{\tdeiaTmp}{\tdeiaFactorPositionY}
  \addtolength{\tdeiaTmp}{0.3cm}
  \pnode(\tdeiaFactorPositionX,\tdeiaTmp){Ff#1}
  
  \addtolength{\tdeiaFactorPositionY}{\tdeiaFactorPositionDy}
}

\newcommand{\tdeiaPaintFactorConnection}[2]
{
  \ncangle[angleA=-90,angleB=180]{-}{Fo#1}{Ff#2}
}

\newcommand{\tdeiaPaintFactorTree}[2]
{
\setlength{\tdeiaFactorPositionDx}{1.0cm}
\setlength{\tdeiaFactorPositionDy}{-0.6cm}
\setlength{\tempX}{#1\tdeiaFactorPositionDx}
\addtolength{\tempX}{5.5cm}
\addtolength{\tempX}{\tdeiaFactorPositionDx}
\setlength{\tempY}{-#2\tdeiaFactorPositionDy}
\addtolength{\tempY}{0.3cm}
\addtolength{\tempY}{-\tdeiaFactorPositionDy}
\tdeiaAdjustpagesize{\tempX}{\tempY}

\begin{figure}
 \centering
 \sf
 \scriptsize
 \begin{pspicture}(0,2\tdeiaFactorPositionDy)(\tempX,\tempY)%\psgrid
  \setlength{\tdeiaFactorPositionY}{-#2\tdeiaFactorPositionDy}
  \addtolength{\tdeiaFactorPositionY}{\tdeiaFactorPositionDy}
  \tdeiaPaintMyFactorAllNodes{}

  \addtolength{\tdeiaFactorPositionY}{\tdeiaFactorPositionDy}
  \rput[bl](\tdeiaFactorPositionDx,\tdeiaFactorPositionY)
  {
    \begin{tabular}{|p{2cm}|p{2.75cm}|p{2.75cm}|}
    \multicolumn{3}{l}{Convención:} \\ \hline
     \rule{0cm}{0.3cm}\cellcolor{gray!25} Nombre & \cellcolor{gray!50}Peso relativo al ambiente&\cellcolor{gray!50}Peso relativo al padre \\ \hline
    \end{tabular}
  }
 \end{pspicture}
 \caption{Árbol de factores ambientales}
 \label{fig:factortree}
\end{figure}
\tdeiaAdjustpagesize{0mm}{0mm}
}

\newcommand{\tdeiaWriteFactor}[5]
{
 \subsection{#2}
 \label{sec:factor#1}
  \begin{center}
   \sf \small
   \begin{tabular}{|*{2}{p{6cm}|}}\hline
    \multicolumn{2}{|p{10cm}|}{\textit{Nombre: }#2} \\ \hline
    \multicolumn{2}{|p{10cm}|}{\textit{Nivel: }#3} \\ 
    \multicolumn{2}{|p{10cm}|}{\textit{Peso respecto al padre: }#4} \\ 
    \multicolumn{2}{|p{10cm}|}{\textit{Peso respecto al nodo raíz: }#5} \\ \hline
    \textit{Nodos padre:} & \textit{Nodos hijo} \\ 
    \begin{itemize}
      \tdeiaWriteMyFactorUp{#1}
    \end{itemize}
    &
    \begin{itemize}
     \tdeiaWriteMyFactorDown{#1}
    \end{itemize}
     \\ \hline
   \end{tabular}
  \end{center}
  \tdeiaDescription{factors}{#1}
}

\newcommand{\tdeiaWriteFactorDown}[2]
{
  \item #2 (Sección \ref{sec:factor#1})
}

\newcommand{\tdeiaWriteFactorUp}[2]
{
  \item #2 (Sección \ref{sec:factor#1})
}

\newcommand{\tdeiaWriteFactorNoParent}{\item Es el nodo raíz. No tiene padres.}
\newcommand{\tdeiaWriteFactorNoChildren}{\item Es un nodo hoja. No tiene hijos.}

%%%%%%%%%%%%%%%%%%%%%%

\newlength{\tdeiaActionPositionX}
\newlength{\tdeiaActionPositionDx}
\newlength{\tdeiaActionPositionY}
\newlength{\tdeiaActionPositionDy}
%\newlength{\tdeiaTmp}

\newcommand{\tdeiaPaintActionNode}[3]
{
  \setlength{\tdeiaActionPositionDx}{1cm}
  \setlength{\tdeiaActionPositionDy}{-0.6cm}
  
  \setlength{\tdeiaActionPositionX}{#3\tdeiaActionPositionDx}
  \rput[bl](\tdeiaActionPositionX,\tdeiaActionPositionY)
  {
    \begin{tabular}{|p{4cm}|}\hline
     \rule{0cm}{0.3cm}\cellcolor{gray!25} \hyperref[sec:action#1]{#2} \\ \hline
    \end{tabular}
  }
 
  \setlength{\tdeiaTmp}{\tdeiaActionPositionX}
  \addtolength{\tdeiaTmp}{0.3cm}
  \pnode(\tdeiaTmp,\tdeiaActionPositionY){Fo#1}

  \setlength{\tdeiaTmp}{\tdeiaActionPositionY}
  \addtolength{\tdeiaTmp}{0.3cm}
  \pnode(\tdeiaActionPositionX,\tdeiaTmp){Ff#1}
  
  \addtolength{\tdeiaActionPositionY}{\tdeiaActionPositionDy}
}

\newcommand{\tdeiaPaintActionConnection}[2]
{
  \ncangle[angleA=-90,angleB=180]{-}{Fo#1}{Ff#2}
}

\newcommand{\tdeiaPaintActionTree}[2]
{
 \setlength{\tdeiaActionPositionDx}{1.0cm}
 \setlength{\tdeiaActionPositionDy}{-0.6cm}
 \setlength{\tempX}{#1\tdeiaActionPositionDx}
 \addtolength{\tempX}{5.5cm}
 \addtolength{\tempX}{\tdeiaActionPositionDx}
 \setlength{\tempY}{-#2\tdeiaActionPositionDy}
 \addtolength{\tempY}{0.3cm}
 \addtolength{\tempY}{-\tdeiaActionPositionDy}
 \tdeiaAdjustpagesize{\tempX}{\tempY}

\begin{figure}
 \centering
 \sf
 \scriptsize
 
 \begin{pspicture}(0,2\tdeiaActionPositionDy)(\tempX,\tempY)%\psgrid
  \setlength{\tdeiaActionPositionY}{-#2\tdeiaActionPositionDy}
  \tdeiaPaintMyActionAllNodes{}
 \end{pspicture}

 \caption{Árbol de acciones del proyecto}
 \label{fig:actiontree}
\end{figure}
 \tdeiaAdjustpagesize{0mm}{0mm}
}

\newcommand{\tdeiaWriteAction}[3]
{
 \subsection{#2}
  \label{sec:action#1}
 \begin{center}
  \sf \small
  \begin{tabular}{|*{2}{p{6cm}|}}\hline
   \multicolumn{2}{|p{10cm}|}{\textit{Nombre: }#2} \\ \hline
   \multicolumn{2}{|p{10cm}|}{\textit{Nivel: }#3} \\ \hline
   \textit{Nodos padre:} & \textit{Nodos hijo} \\ 
   \begin{itemize}
     \tdeiaWriteMyActionUp{#1}
   \end{itemize}
   &
   \begin{itemize}
    \tdeiaWriteMyActionDown{#1}
   \end{itemize}
    \\ \hline
  \end{tabular}
 \end{center}
 
 \tdeiaDescription{actions}{#1}
}

\newcommand{\tdeiaWriteActionDown}[2]
{
  \item #2 (Sección \ref{sec:action#1})
}

\newcommand{\tdeiaWriteActionUp}[2]
{
  \item #2 (Sección \ref{sec:action#1})
}

\newcommand{\tdeiaWriteActionNoParent}{\item Es el nodo raíz. No tiene padres.}
\newcommand{\tdeiaWriteActionNoChildren}{\item Es un nodo hoja. No tiene hijos.}

%%%%%%%%%%%%%%%%%%%%%%%%%%
\newcommand{\tdeiaWriteInputsHead}[6]
{
 \caption{Valoración de las propiedades del efecto `#2' y de su importancia}
 \label{tab:effectInputs#1} \\ \hline 
 \multicolumn{4}{|p{17cm}|}{\textit{Nombre:} #2} \\ \hline
 \multicolumn{4}{|p{17cm}|}{\textit{Factor:} #4 (Sección \ref{sec:factor#3})} \\ \hline
 \multicolumn{4}{|p{17cm}|}{\textit{Action:} #6 (Sección \ref{sec:action#5})} \\ \hline 
 \multicolumn{4}{|p{17cm}|}{\textit{Descripción del efecto:}} \\  
 \multicolumn{4}{|p{17cm}|}{\tdeiaDescription{effects}{#1}} \\ \hline 
 \textbf{Propiedad} & \textbf{Tipo y Valor} & \textbf{Justificación} & \textbf{Gráfico} \\ \hline \endfirsthead
 \hline
 \multicolumn{4}{|p{17cm}|}{\textit{Nombre:} #2} \\ \hline
 \textbf{Propiedad} & \textbf{Tipo y Valor} & \textbf{Justificación} & \textbf{Gráfico} \\ \hline  \endhead
}

\newcommand{\tdeiaInputType}[2]
{
  \ifthenelse{\equal{#1}{number}}{Número}{}
  \ifthenelse{\equal{#1}{interval}}{Intervalo}{}
  \ifthenelse{\equal{#1}{fuzzy_number}}{Número difuso}{}
  \ifthenelse{\equal{#1}{label}}
  {
    \ifthenelse{\equal{#2}{0}}{Etiqueta}{Etiqueta con modificador}
  }{}
}

\newcommand{\tdeiaWriteInput}[6]
{
  #2 (cuadro \ref{tab:effect_propierty_#1}) &
  \begin{minipage}{3cm}
    \tdeiaInputType{#4}{#5}: #6
  \end{minipage}
 &
  \begin{minipage}{4cm}
    \tdeiaDescription{inputs}{#3}    
  \end{minipage}
 &
  \scalebox{0.6}
  {
   \begin{pspicture}(0,0)(10,3)%\psgrid
    \rput[bl](1,0){\tdeiaPaintMyInput{#3}}
   \end{pspicture}
  }
  \\ \hline
}

\newcommand{\tdeiaWriteImportance}[7]
{
  \textbf{Importancia} &
  \textbf{Valor} &
  \textbf{Análisis} &
  \textbf{Gráfico} \\ \hline
  cuadro \ref{tab:importanceVar} &
  \begin{minipage}{3cm}
    \textit{Interpretación:} \textbf{#4}\\
    \textit{Valor:} \textbf{#6}\\
    \textit{Ambigüedad:} \textbf{#7} \\ \\
    \textit{#5}\\
  \end{minipage}
 &
   #3
 &
  \scalebox{0.6}
  {
   \begin{pspicture}(0,0)(10,3)%\psgrid
    \rput[bl](1,0){\tdeiaPaintMyImportance{#2}}
   \end{pspicture}
  }
  \\ \hline
}

\newenvironment{tdeiaWriteEffectInputs}[2]
{
 \begin{landscape}
  \sf
  \scriptsize
  \begin{longtable}{|p{2cm}|p{3cm}|p{5cm}|p{7cm}|}
}
{
  \end{longtable}
 \end{landscape}
 \clearpage
}

%%%%%%%%%%% MATRIX

\newlength{\cellX}
\newlength{\cellY}
\newlength{\cellXH}
\newlength{\cellYH}
\newlength{\tableX}
\newlength{\tableY}


\newcommand{\tablePage}[2]
{
  \setlength{\cellX}{1.25cm}
  \setlength{\cellXH}{2.0cm}
  \setlength{\cellY}{0.5cm}
  \setlength{\cellYH}{0.5cm}
  \setlength{\tableX}{#1\cellX}
  \addtolength{\tableX}{\cellXH}
  \addtolength{\tableX}{\cellX}
  \setlength{\tableY}{#2\cellY}
  \addtolength{\tableY}{\cellYH}
  \addtolength{\tableY}{\cellY}
  %%% figure
  \addtolength{\tableY}{4cm}
  %%%
  \ifthenelse{\tableX > \tableY \AND \tableX > \textwidth}
  {
    \tdeiaAdjustpagesize{\tableX}{\tableY}
  }
  {
    \tdeiaAdjustpagesize{\tableX}{\tableY}
  }
}

\newenvironment{tdeiaMatrix}[7]
{
  \def\fooVID{#3}
  \def\fooVTY{#4}
  \setlength\tabcolsep{0cm}
  \ifthenelse{\tableX > \tableY \AND \tableX > \textwidth}
  {
    \begin{sidewaystable} 
  }
  {
    \begin{table} 
  }
  \centering
   \caption[#5]{#6}
  \label{#7}
  \tiny \sf
  \begin{tabular}{|p{\cellXH}|*{#1}{p{\cellX}|}p{\cellX}|}\hline
}
{
  \end{tabular}
  \ifthenelse{\equal{\fooVID}{0}}
  {
  }
  {
    \\
    \begin{pspicture}(0,-1)(8,3.0)%\psgrid
      \tdeiaPaintMyGradBlock{\fooVTY}
      \tdeiaPaintMyVariable{\fooVID}   
    \end{pspicture}
  }

  \ifthenelse{\tableX > \tableY \AND \tableX > \textwidth}
  {
    \end{sidewaystable}
  }
  {
    \end{table}
  }
}


\newlength{\dxBlock}
\newlength{\dyBlock}

\newcommand{\tdeiaPaintSubBlockGrad}[4]
{
  \setlength{\dxBlock}{5cm}
  \setlength{\dyBlock}{-0.5cm}
  \definecolor{colbegin}{HTML}{#1}
  \definecolor{colend}{HTML}{#2}
  \psframe[fillstyle=gradient,gradbegin=colbegin,gradend=colend,gradmidpoint=1.0,gradangle=90,linewidth=0pt,linestyle=none]
  (#3\dxBlock,0)(#4\dxBlock,\dyBlock)
}


\newcommand{\tdeiaMatrixEmptyCell}{}

\newcommand{\tdeiaMatrixHeaderColCell}[1]{\rule{0cm}{\cellY}\textbf{#1}}
\newcommand{\tdeiaMatrixHeaderTotalCell}[1]{\textbf{#1}}
\newcommand{\tdeiaMatrixHeaderRowCell}[1]{\rule{0cm}{\cellY}\textbf{#1}}
\newcommand{\tdeiaMatrixHeaderRowCellMulti}[2]
{
% \multirow{#2}{\cellXH}{ \rule{0cm}{#2\cellY}\textbf{#1}}
 \rule{0cm}{\cellY}\textbf{#1}
}

\newcommand{\tdeiaMatrixRowTotalCell}[1]{\rule{0cm}{\cellY}\textbf{#1}}
\newcommand{\tdeiaMatrixCellContent}[1]{#1}
\newcommand{\tdeiaIndividualCellContent}[1]{#1}
\newcommand{\tdeiaColorCell}[1]{\cellcolor[HTML]{#1}\rule{0cm}{0cm}}


\newcommand{\tdeiaTableSectionCellType}[1]{\part{#1}}
\newcommand{\tdeiaTableSectionType}[1]{\chapter{#1}}
\newcommand{\tdeiaTableSectionAgg}[1]{}%{\section{#1}}


%%%%%%%%%%%%%%%%%%%%%%%%%%%%%%
\newenvironment{tdeiaWriteMyImportanceSummary}
{
  \begin{landscape}
   \sf
   \scriptsize
   \begin{longtable}{|p{3cm}|p{3cm}|p{3cm}|p{2cm}|p{4cm}|}
   \caption{Resumen de efectos individuales y sus importancias} 
   \label{tab:effectSummary}
   \\ \hline
   \textbf{Factor} & \textbf{Acción} & \textbf{Efecto} & \textbf{Importancia} & \textbf{Análisis} \\ \hline \endhead
}
{
   \end{longtable}
  \end{landscape}
}

\newcommand{\tdeiaWriteMyImportanceInSummary}[9]
{
#2 (Sección \ref{sec:factor#1}) & 
#4 (Sección \ref{sec:action#3}) & 
#9 (Cuadro \ref{tab:effectInputs#5}) &
 #6 ($#8$) & 
 #7 
 \\ \hline
}